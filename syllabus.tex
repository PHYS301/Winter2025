% Options for packages loaded elsewhere
\PassOptionsToPackage{unicode}{hyperref}
\PassOptionsToPackage{hyphens}{url}
\PassOptionsToPackage{dvipsnames,svgnames,x11names}{xcolor}
%
\documentclass[
  letterpaper,
  DIV=11,
  numbers=noendperiod]{scrartcl}

\usepackage{amsmath,amssymb}
\usepackage{iftex}
\ifPDFTeX
  \usepackage[T1]{fontenc}
  \usepackage[utf8]{inputenc}
  \usepackage{textcomp} % provide euro and other symbols
\else % if luatex or xetex
  \usepackage{unicode-math}
  \defaultfontfeatures{Scale=MatchLowercase}
  \defaultfontfeatures[\rmfamily]{Ligatures=TeX,Scale=1}
\fi
\usepackage{lmodern}
\ifPDFTeX\else  
    % xetex/luatex font selection
\fi
% Use upquote if available, for straight quotes in verbatim environments
\IfFileExists{upquote.sty}{\usepackage{upquote}}{}
\IfFileExists{microtype.sty}{% use microtype if available
  \usepackage[]{microtype}
  \UseMicrotypeSet[protrusion]{basicmath} % disable protrusion for tt fonts
}{}
\makeatletter
\@ifundefined{KOMAClassName}{% if non-KOMA class
  \IfFileExists{parskip.sty}{%
    \usepackage{parskip}
  }{% else
    \setlength{\parindent}{0pt}
    \setlength{\parskip}{6pt plus 2pt minus 1pt}}
}{% if KOMA class
  \KOMAoptions{parskip=half}}
\makeatother
\usepackage{xcolor}
\setlength{\emergencystretch}{3em} % prevent overfull lines
\setcounter{secnumdepth}{5}
% Make \paragraph and \subparagraph free-standing
\ifx\paragraph\undefined\else
  \let\oldparagraph\paragraph
  \renewcommand{\paragraph}[1]{\oldparagraph{#1}\mbox{}}
\fi
\ifx\subparagraph\undefined\else
  \let\oldsubparagraph\subparagraph
  \renewcommand{\subparagraph}[1]{\oldsubparagraph{#1}\mbox{}}
\fi


\providecommand{\tightlist}{%
  \setlength{\itemsep}{0pt}\setlength{\parskip}{0pt}}\usepackage{longtable,booktabs,array}
\usepackage{calc} % for calculating minipage widths
% Correct order of tables after \paragraph or \subparagraph
\usepackage{etoolbox}
\makeatletter
\patchcmd\longtable{\par}{\if@noskipsec\mbox{}\fi\par}{}{}
\makeatother
% Allow footnotes in longtable head/foot
\IfFileExists{footnotehyper.sty}{\usepackage{footnotehyper}}{\usepackage{footnote}}
\makesavenoteenv{longtable}
\usepackage{graphicx}
\makeatletter
\def\maxwidth{\ifdim\Gin@nat@width>\linewidth\linewidth\else\Gin@nat@width\fi}
\def\maxheight{\ifdim\Gin@nat@height>\textheight\textheight\else\Gin@nat@height\fi}
\makeatother
% Scale images if necessary, so that they will not overflow the page
% margins by default, and it is still possible to overwrite the defaults
% using explicit options in \includegraphics[width, height, ...]{}
\setkeys{Gin}{width=\maxwidth,height=\maxheight,keepaspectratio}
% Set default figure placement to htbp
\makeatletter
\def\fps@figure{htbp}
\makeatother

\KOMAoption{captions}{tableheading}
\makeatletter
\@ifpackageloaded{caption}{}{\usepackage{caption}}
\AtBeginDocument{%
\ifdefined\contentsname
  \renewcommand*\contentsname{Table of contents}
\else
  \newcommand\contentsname{Table of contents}
\fi
\ifdefined\listfigurename
  \renewcommand*\listfigurename{List of Figures}
\else
  \newcommand\listfigurename{List of Figures}
\fi
\ifdefined\listtablename
  \renewcommand*\listtablename{List of Tables}
\else
  \newcommand\listtablename{List of Tables}
\fi
\ifdefined\figurename
  \renewcommand*\figurename{Figure}
\else
  \newcommand\figurename{Figure}
\fi
\ifdefined\tablename
  \renewcommand*\tablename{Table}
\else
  \newcommand\tablename{Table}
\fi
}
\@ifpackageloaded{float}{}{\usepackage{float}}
\floatstyle{ruled}
\@ifundefined{c@chapter}{\newfloat{codelisting}{h}{lop}}{\newfloat{codelisting}{h}{lop}[chapter]}
\floatname{codelisting}{Listing}
\newcommand*\listoflistings{\listof{codelisting}{List of Listings}}
\makeatother
\makeatletter
\makeatother
\makeatletter
\@ifpackageloaded{caption}{}{\usepackage{caption}}
\@ifpackageloaded{subcaption}{}{\usepackage{subcaption}}
\makeatother
\ifLuaTeX
  \usepackage{selnolig}  % disable illegal ligatures
\fi
\usepackage{bookmark}

\IfFileExists{xurl.sty}{\usepackage{xurl}}{} % add URL line breaks if available
\urlstyle{same} % disable monospaced font for URLs
\hypersetup{
  pdftitle={Syllabus},
  colorlinks=true,
  linkcolor={blue},
  filecolor={Maroon},
  citecolor={Blue},
  urlcolor={Blue},
  pdfcreator={LaTeX via pandoc}}

\title{Syllabus}
\author{}
\date{}

\begin{document}
\maketitle

\renewcommand*\contentsname{Table of contents}
{
\hypersetup{linkcolor=}
\setcounter{tocdepth}{3}
\tableofcontents
}
\section{Classroom expectations}\label{classexpectations}

\subsection{What you can expect from me}

\begin{itemize}
\tightlist
\item
  I will stay home if I am feeling sick and make arrangements to deliver
  the course material
\item
  I will work with you to arrange accommodations when you need them
\item
  I will respect your time by starting and ending class on time
\item
  I will answer your questions thoughtfully, and if I don't know the
  answer, I will follow up in a timely manner
\item
  I will embrace who you are as whole people
\item
  I will model respect, openness, and engagement, and foster a
  supportive and inclusive environment
\item
  I will be honest when I make mistakes, because failure is part of
  growing
\end{itemize}

\subsection{What I expect from you}

\begin{itemize}
\tightlist
\item
  That you will stay home if you are sick and contact me via email to
  arrange accommodations
\item
  That you genuinely attempt to engage with the course
\item
  That you ask questions if you are confused (you may do this privately
  -- there is no obligation to ask during class hours)
\item
  That you communicate with me when you have problems that interfere
  with your ability to engage with the coursework
\item
  That you treat your peers with respect and openness, and that you
  participate in creating an inclusive, supportive, and engaged
  classroom
\end{itemize}

\subsection{What is not expected}

\begin{itemize}
\tightlist
\item
  Perfection. Ever. It's a myth.
\item
  That you will `sit still' or ask for permission to leave the classroom
  to go to the bathroom or if you just need a minute.
\item
  That everyone will learn in the same way. You do not have to match
  some ``model student'' to do well in this class
\end{itemize}

\section{Team-Based Learning}\label{tbl}

I am using a variation of team-based learning for this class, in order
to cultivate a community-minded classroom, encourage a growth mindset,
and build group work skills. Here is how this will work:

We will have three modules, and you will work in a team of 4-5 students
for each module. Your team will work on problems in class, discuss the
content, and turn in a weekly problem that you solve together. At the
end of each module, you will provide feedback on your teammates and on
your own work, and then we will shuffle the groups, so you will have
three different teams over the course of the semester.

Not all of the work will be in groups. You will also have individual
quizzes and homework. You can work with other students on your homework,
but you must submit and self-assess your own homework.

I have tried to balance the class so that there is a mix of individual
and group work, which I hope will allow everyone to get something
meaningful out of the course.

\section{Assignments and Grading}\label{assignments}

Assignments fall into ``bundles,'' which contribute to your grade in
specific ways. Your performance on each bundle determines your rough
letter grade (full letters). Beyond that, you can achieve grade boosts,
which round your grade up, e.g.~from a B to a B+, or a B+ to an A-.

You can learn more about each bundle below:

\hyperref[pcqs]{PCQs}

\hyperref[whws]{WHWs}

\hyperref[groupwork]{Group problems}

\hyperref[quizzes]{Weekly quizzes}

\hyperref[participation]{Community engagement}

\subsection{Grading Scheme}\label{gradingscheme}

\subsection{Table of letter grades}\label{table-of-letter-grades}

\begin{longtable}[]{@{}
  >{\raggedright\arraybackslash}p{(\columnwidth - 10\tabcolsep) * \real{0.0864}}
  >{\raggedright\arraybackslash}p{(\columnwidth - 10\tabcolsep) * \real{0.1728}}
  >{\raggedright\arraybackslash}p{(\columnwidth - 10\tabcolsep) * \real{0.1481}}
  >{\raggedright\arraybackslash}p{(\columnwidth - 10\tabcolsep) * \real{0.2160}}
  >{\raggedright\arraybackslash}p{(\columnwidth - 10\tabcolsep) * \real{0.2160}}
  >{\raggedright\arraybackslash}p{(\columnwidth - 10\tabcolsep) * \real{0.1605}}@{}}
\caption{Letter grades}\tabularnewline
\toprule\noalign{}
\begin{minipage}[b]{\linewidth}\raggedright
Letter grade
\end{minipage} & \begin{minipage}[b]{\linewidth}\raggedright
Pre-Class Questions (PCQs)
\end{minipage} & \begin{minipage}[b]{\linewidth}\raggedright
Weekly Homeworks (WHWs)
\end{minipage} & \begin{minipage}[b]{\linewidth}\raggedright
Group Problems (x12)
\end{minipage} & \begin{minipage}[b]{\linewidth}\raggedright
Quizzes (x10)
\end{minipage} & \begin{minipage}[b]{\linewidth}\raggedright
Community Engagement
\end{minipage} \\
\midrule\noalign{}
\endfirsthead
\toprule\noalign{}
\begin{minipage}[b]{\linewidth}\raggedright
Letter grade
\end{minipage} & \begin{minipage}[b]{\linewidth}\raggedright
Pre-Class Questions (PCQs)
\end{minipage} & \begin{minipage}[b]{\linewidth}\raggedright
Weekly Homeworks (WHWs)
\end{minipage} & \begin{minipage}[b]{\linewidth}\raggedright
Group Problems (x12)
\end{minipage} & \begin{minipage}[b]{\linewidth}\raggedright
Quizzes (x10)
\end{minipage} & \begin{minipage}[b]{\linewidth}\raggedright
Community Engagement
\end{minipage} \\
\midrule\noalign{}
\endhead
\bottomrule\noalign{}
\endlastfoot
D & Earn 75\% of points & Earn 75\% of points & At least M on 6 & At
least M on 5 & At least M on 1 module \\
C & Earn 75\% of points & Earn 75\% of points & At least M on 9 & At
least M on 8 & At least M on 2 modules \\
B & Earn 75\% of points & Earn 75\% of points & At least M on all 12,
plus 2 Es & At least M on all 12, plus 2 Es & At least M on all 3
modules \\
A & Earn 75\% of points & Earn 75\% of points & At least M on all 12,
plus 4 Es & At least M on all 12, plus 4 Es & At least M on all 3
modules \\
\end{longtable}

\subsection{Bundle visualization}\label{bundle-visualization}

\includegraphics{images/PHYS211_GradingBundles.png}

\subsection{Grading scales}\label{grading-scales}

\subsubsection{Points}

Homeworks will be graded on a points scale: each WHW is worth 100
points, and you can earn a portion of those points by successfully and
thoroughly solving the problem on the homework.

\subsubsection{Completion}

Pre-class questions will be graded on completion. What this means is
that you must give a good-faith attempt at the problem, but do not need
to get it correct. If you turn nothing in, or turn in something that is
incomplete or unrelated, you will get no credit for the PCQ.

\subsubsection{E/M/U/N}

Unit tests, individual final portfolio problems, and the two problem
projects will be graded on the following scale:

\begin{itemize}
\tightlist
\item
  E: excellent -- this is a thorough and correct answer that
  demonstrates excellent understanding of the concepts and makes proper
  use of the mathematical skills expected in this class.
\item
  M: meets expectations -- this is an answer that demonstrates solid
  understanding of the concepts but perhaps includes some small
  mathematical errors or minor conceptual mistakes.
\item
  U: unsatisfactory / does not meet expectations -- this is an answer
  that applies the concepts incorrectly, misunderstands the point of the
  question, does not complete the question, fails to follow directions,
  and/or contains significant mathematical errors
\item
  N: no submission -- if you turn nothing in, you will receive an N.
\end{itemize}

E and M are considered passing grades. To get an A or B in this class,
you must achieve some Es (see the \hyperref[gradingscheme]{grading
scheme})

U and N are considered failing grades.

\subsection{Grading bundles}\label{grading-bundles}

\subsubsection{PCQs}

In order to get the most out of class, you need to prepare. To do this,
most days we will have a pre-class question due in Lyceum, which you
must submit prior to class start. These are graded on completion, which
simply means that you demonstrate a good-faith effort to solve the
problem. You do not have to get it right, and we will go over the
problem during class.

Given the purpose of the PCQs, I will not offer extensions, except in
case of a major medical or family emergency that causes you to miss
class.

\subsubsection{WHWs}

Regular problem-solving is very important to developing confidence and
skill in this content, so every week you will have a short weekly
problem set. You will turn in the weekly problem set by scanning and
uploading as a PDF to Moodle. These are due by 8PM each Wednesday.

5\% of each homework is a reflection question, which means that simply
responding to it will earn you the points, and each homework is graded
out of 100 points. To pass this bundle (which is required to pass the
class), you must earn 60\% of all the available homework points
throughout the semester. In general, the majority of questions will be
on material covered on or before the Friday before the WHW is due, but
there may be a question on material covered the Monday before the WHW is
due. This means you should be able to complete most of the homework
before Monday's class!

You may work with classmates on the homework, but please be sure to give
credit in your reflection question.

You may request extensions on the WHWs using the
\href{https://forms.gle/eFx7y7FoSdoukKGC6}{extension form}.

\subsubsection{Final Portfolio}

The idea for this final portfolio assignment is to demonstrate what you
have learned in the course and to create something for which you can
take ownership and feel proud. We will get a lot of practice with
problem solving this semester, and this is a chance to go back and
curate some of what you have done.

The intended audience for this assignment is a fellow student who had
not previously seen the homework questions you have chosen, but is
curious to know how to solve them. For example, you could imagine
yourself at the beginning of the semester, or a classmate earlier this
semester who was unsure how to approach a problem. Would that intended
audience be able to understand your approach and your logic? Would they
feel empowered to learn based on what you have created for them?

The assignment will involve redoing 8 homework questions (one from each
homework assignment) in a manner that clearly presents the logic and
reasoning behind your answers. It will be useful to restate the problem
in your own words, clearly identifying known and unknown quantities. In
many cases, it will be useful to draw a picture of the situation and to
establish a clear coordinate system. You will then want to describe and
demonstrate your approach to solving the problem. Once you arrive at an
answer, it will also be helpful to reflect on the implications of that
answer. This could include a consideration of how the answer depends on
certain variables, or how the answer might be different for a slightly
different situation.

Each problem is graded on an \hyperref[emun]{E/M/U/N scale}, and an E
and M are both considered passing grades.

You may request an extension on the final portfolio using the
\href{https://forms.gle/eFx7y7FoSdoukKGC6}{extension form}. Please note,
however, that deadline cannot be extended beyond the end of final exam
period.

\subsubsection{Unit tests}

There will be three unit tests during class throughout this semester.
The tests are cumulative, but they will largely focus on the more recent
material. These tests will occur in class, and you may bring a
single-sided 8.5''x11'' sheet of paper with equations and notes -
hand-written (please speak to me if you need to type the sheet and we
will come up with a plan). This paper will be turned in with your test.

These are graded on an \hyperref[emun]{E/M/U/N scale}, and you may
retake each unit test up to two times during the scheduled make-up test
days. In order to retake a unit test, you must first correct your first
unit test and complete a short reflection in a
\href{https://forms.gle/rfhtp2ALtBY8JFGT9}{Google form}.

\subsubsection{Problem Projects}

The student becomes the teacher\ldots{} you will have two opportunities
to write a physics problem similar to your homework problems. You must:

\begin{itemize}
\tightlist
\item
  select at least one concept each from two different concept
  \href{coursecontent.qmd\#concepts}{groups}
\item
  select one \href{coursecontent.qmd\#mathtools}{mathematical tool} to
  highlight
\item
  write a physics problem that tests understanding of those concepts and
  uses the math tool
\item
  write up a detailed solution to that problem using the
  \href{coursecontent.qmd\#fourstepmethod}{four-step problem solving
  method}
\item
  explain how your problem tests understanding for the concepts you
  selected
\end{itemize}

These are graded on an \hyperref[emun]{E/M/U/N scale}, and you may use
each problem project score to replace one of the following:

\begin{itemize}
\tightlist
\item
  your WHW score for the semester (an M or higher gives you full credit)
\item
  your PCQ score for the semester (an M or higher gives you full credit)
\item
  one unit test score
\item
  one final portfolio problem
\end{itemize}

I will always replace the grade item that most benefits you, and will
never replace a better score with a worse one. The Problem Projects can
only help you.

If you want to get a B in the class, you have to at least try a problem
project, but you do not have to pass. To get an A in the class, you have
to pass at least one problem project.

If you do two problem projects, they must be different problems (you
can't revise your first problem to get credit for the second problem.)

You may request extensions on the problem projects using the
\href{https://forms.gle/eFx7y7FoSdoukKGC6}{extension form}. Please note,
however, that the final problem project cannot be extended beyond the
end of final exam period.

\subsection{Grading boosts}\label{grading-boosts}

\subsubsection{Effort Boost}

If you complete 85\% of pre-class questions, turn in every WHW
assignment, turn in a complete portfolio, and attempt at least one
problem project, you will gain a grade round up for effort, regardless
of the outcome of your work.

\subsubsection{Problem-Solver Boost}

If you earn at least 90\% of the available homework points, you will
gain a grade round up for demonstrating strong, consistent
problem-solving skills.

\subsubsection{Growth Boost}

If you demonstrate consistent improvement in your work across the
semester, I reserve the right to round your grade up.

\subsubsection{Metacognition Boost}

In order to get this boost, you must do all of the following:

\begin{itemize}
\tightlist
\item
  Fill out the \href{https://forms.gle/K3kqhLWRtxuAd45U8}{mid-semester
  survey} before class on 10/25
\item
  Attend the SASC workshop during class on 10/25
\item
  Attend at least one SASC Physics 211 help session and
  \href{https://forms.gle/qudMKQ4FJX76pBef9}{fill out a response} about
  your experience
\item
  Set a
  \href{https://docs.google.com/forms/d/e/1FAIpQLSfMTioS5vSv3KYcqxJR8C2WfCp9XX4KIKFMHYWWeeLbHjmJjw/viewform}{SMART
  goal} (this will be evaluated on whether it meets the criteria) for a
  study skill to improve. You must do this before November 15.
\item
  \href{https://forms.gle/Cenw9TjhTMgz2qod7}{Reflect on your goal} at
  the end of the semester
\end{itemize}

\section{Deadlines and Extensions}\label{deadlines}

If you need an extension, you may request one using
\href{https://forms.gle/eFx7y7FoSdoukKGC6}{this form}. I recognize that
things come up and you may require flexibility at some point in the
semester. Please feel free to reach out to me directly if you are
struggling to meet a deadline. I want to support you and make sure you
have the best possible chance for success in this class, and the only
way I can help is if you communicate with me.

In general, I am happy to be flexible. Please note, however, that some
assignments will have stricter deadlines. These assignments include the
PCQs and the final portfolio and final problem project, and the nature
of the deadlines is discussed in their descriptions above.



\end{document}
